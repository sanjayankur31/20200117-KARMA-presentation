%% Ankur Sinha

%% packages %%
% support for coloured text
\usepackage{color}
% IPA
\usepackage{tipa}
\usepackage[scale=2]{ccicons}
\usepackage{amssymb}
\usepackage{tikz}
\usetikzlibrary{mindmap, arrows.meta, positioning, arrows}
\usepackage{pgfplots}
% Define the colours we use for E and I in all graphs
\definecolor{SinhaBlueE}{HTML}{3b4cc0}
\definecolor{SinhaRedI}{HTML}{f7a789}
\pgfmathdeclarefunction{gaussnew}{4}{%nu, eta, eps, omega
  \pgfmathparse{(#1*((2*exp(-(((x-((#2+#3)/2))/((#2-#3)/(2*sqrt(-ln(#4/2)))))^2))) -#4))}%chktex 36
}
\usepackage{jneurosci}
\usepackage{subfig}
\usepackage[T1]{fontenc}
\usepackage[utf8]{inputenc}
\usepackage[style=nature,backend=biber,autocite=footnote]{biblatex}
\addbibresource{/home/asinha/Documents/01_Readables/00_research_papers/masterbib.bib}
% Use opensans
% \usepackage[default,scale=0.95]{opensans}
\usepackage[sfdefault]{roboto}
% for strike through
\usepackage[normalem]{ulem}
% links, urls, refs
\definecolor{links}{HTML}{2A1B81}
% Fedora blue for the theme
\definecolor{FedoraBlue}{HTML}{2A1B81}
\usepackage{hyperref}
\hypersetup{colorlinks,linkcolor=Green,urlcolor=links}
% graphics
\usepackage{graphicx}
% algorithm
\usepackage{algorithmic}
\usepackage{textcomp}
\usepackage{wrapfig}
\usepackage{textgreek}
\usepackage{euler}
\usepackage{csquotes}

% beamer theme
% use defaults for theme
\usetheme[numbering=fraction]{metropolis}
\usefonttheme[onlymath]{serif}
\setbeamerfont{footnote}{size=\tiny}
\setbeamerfont{caption}{size=\tiny}
\setbeamercolor{alerted text}{fg=Green}
\setbeamerfont{note page}{size=\small}

% Not needed in metropolis, but in general footnote citation fixes: https://tex.stackexchange.com/questions/44217/how-can-i-stop-footcite-from-hijacking-my-beamer-columns
% how to use multiple references to the same footnote: https://tex.stackexchange.com/questions/27763/beamer-multiple-references-to-the-same-footnote

% Disable footnoterule
\renewcommand{\footnoterule}{}

%% title %%
\title{Investigating structural plasticity in brain networks using computational modelling}
\author[Ankur Sinha]{Ankur Sinha\\PhD candidate\\Biocomputation Research Group\\University of Hertfordshire.}
\date{17/01/2020}

%% document begins %%
\begin{document}


% title frame %%
\begin{frame}
  \titlepage{}
\end{frame}

%% Three slides for 5 minutes seems good
%% So, 30 slides at most for 50 minutes
\section{Context: what and why?}
\begin{frame}[c]{The plastic---but stable---brain: Hebbian/Homeostatic plasticity}
  \pause{}
  \begin{itemize}
    \item \enquote{Neurons that fire together, wire together.}\footnotemark[1]{}
      \pause{}
    \item \enquote{The more things change, the more they stay the same.}\footnotemark[2]{}
  \end{itemize}
  \footnotetext[1]<2->{\fullcite{Hebb1949}}
  \footnotetext[2]<3->{\fullcite{Turrigiano1999}}
\end{frame}
\begin{frame}[c]{Synaptic plasticity: the popular plasticity}
  \begin{itemize}
    \item changes in efficacy of \alert{existing} synapses,
      \pause{}
    \item changes in structure are ignored\footnote[1]{Even though structural changes in spines and boutons underlie modulation of synaptic efficacy.}.
  \end{itemize}
\end{frame}
\begin{frame}[c]{What underlies large scale reorganisation?}
  \begin{itemize}
    \item \fullcite{Rasmusson1982}
  \pause{}
  \scriptsize{
      \item \fullcite{Wall1984}
      \item \fullcite{Merzenich1984}
      \item \fullcite{Calford1988}
      \item \fullcite{Heinen1991}
      \item \fullcite{Rajan1993}
      }
  \end{itemize}
\end{frame}
\begin{frame}[c]{Two possible theories}
  \begin{itemize}
    \item \enquote{unmasking} of pre-existing synaptic connections,
      \pause{}
    \item formation of new synapses (\alert{structural plasticity}).
  \end{itemize}
  \footnotetext[1]{\fullcite{Rasmusson1982}}
\end{frame}
\begin{frame}[c]{Imaging confirms structural plasticity in lesion studies}
    \begin{itemize}
      \item \fullcite{Darian-Smith1994}
        \pause{}
        \footnotesize{
      \item \fullcite{Florence1998}
      \item \fullcite{Keck2008}
      \item \fullcite{Keck2011}
      \item \fullcite{Marik2014}
      }
    \end{itemize}
\end{frame}
\begin{frame}[c]{Also confirms structural plasticity in the unlesioned adult brain}
    \begin{itemize}
      \item \fullcite{Holtmaat2005}
        \pause{}
        \footnotesize{
      \item \fullcite{Stettler2006}
      \item \fullcite{Marik2010}
      \item \fullcite{Chen2012}
      \item \fullcite{Villa2016}
      }
    \end{itemize}
\end{frame}
\begin{frame}[c]{So:}
  \begin{itemize}
    \item not only do the strengths of existing synapses change,
    \item \alert{whole synapses are formed and removed.}
      \pause{}
    \item How? Why?
  \end{itemize}
\end{frame}
\begin{frame}[c]{Aim}
  \begin{center}
    \textbf{Simulate a computational model of peripheral lesioning to study the reorganisation process.}
  \end{center}
  \pause{}
  \begin{itemize}
    \item A computational model allows us to:
      \begin{itemize}
        \item investigate every entity in the network: variables from neurons, their neurites, and all synapses,
        \item modify any parameters to analyse changes in network behaviour: neuronal parameters, synaptic parameters,
        \item run multiple analyses in parallel,
        \item do it in less time than biological experiments\footnote[1]<2->{Simulations still take a week each, but that's still faster than a multi-month laboratory experiment.}.
      \end{itemize}
  \end{itemize}
\end{frame}
\section{Methods: How?}
\begin{frame}[c]{Peripheral lesion protocol I:\ topographic mapping}
  \note[item]{The protocol is pretty standard. Here, for a study in the visual cortex, the retinal field of a rat or a mouse is mapped.}
  \begin{columns}
    \begin{column}{0.5\textwidth}
      \centering
      \includegraphics[width=0.8\textwidth]{99_images/keck-1-1a}%chktex 8
    \end{column}
    \begin{column}{0.5\textwidth}
      \centering
      \includegraphics[width=0.8\textwidth]{99_images/keck-1-1c}%chktex 8
    \end{column}
  \end{columns}
  \footnotetext[1]{\fullcite{Keck2008}}
\end{frame}
\begin{frame}[c]{Peripheral lesion protocol II:\ after peripheral lesion}
  \note[item]{Then, a part of the retina is lesioned. This cuts off inputs to a part of the visual cortex, as shown in the first figure. This forms the Lesion Projection Zone (LPZ). By repeated imaging of the region over months, the reorganisation of the network is tracked.}
  \note[item]{Other lesion studies use similar methods: digit removal, whisker trimming, and so on---anything that cuts off projecting activity on to a set of neurons.}
  \begin{figure}[h]
    \centering
    \includegraphics[width=\textwidth]{99_images/keck-1-2c}%chktex 8
  \end{figure}
  \begin{itemize}
    \item Dotted region encloses the Lesion Projection Zone (LPZ)
    \item Inward \enquote{repair}.
  \end{itemize}
    \footnotetext[1]{\fullcite{Keck2008}}
\end{frame}
\begin{frame}[c]{Data gathered from these experiments: summary}
  \note[item]{So, if this a simple schematic, of the regions around the LPZ, this is what we know from these studies.}
  \begin{columns}
    \begin{column}{0.3\textwidth}
      \centering
      \input{99_images/regions.tex}
    \end{column}
    \pause{}
    \begin{column}{0.6\textwidth}
      \begin{itemize}
        \item Inward repair of network.
        \item Massive disinhibition in the LPZ\@.
        \item Gradual \alert{ingrowth of excitatory synapses} from the peri-LPZ to the LPZ\@.
        \item Gradual \alert{outgrowth of inhibitory synapses} from the LPZ to the peri-LPZ\@.
      \end{itemize}
    \end{column}
  \end{columns}
\end{frame}
\begin{frame}[c]{Physiological cortical spiking network model}
  \begin{figure}[h]
    \def\svgwidth{0.7\textwidth}
    \input{99_images/schematic.tex}
  \end{figure}
  \footnotetext[1]{\fullcite{Vogels2011}}
\end{frame}
\begin{frame}[t]{Neuron model: 8000 E, 2000 I neurons}
  \begin{columns}
    \begin{column}{0.6\textwidth}
      \begin{itemize}
        \item Point \enquote{leaky integrate and fire neurons}\footnotemark[1]{}, 
        \item Bear neurites (\(z\)).
      \end{itemize}
    \end{column}
    \begin{column}{0.4\textwidth}
      \input{99_images/neuron-str-p.tex}
    \end{column}
  \end{columns}
  \footnotetext[1]{\fullcite{Meffin2004}}
\end{frame}
\begin{frame}[t]{Modelling synapse formation and removal}
  \begin{columns}
    \begin{column}{0.6\textwidth}
      \begin{itemize}
        \item \(z^E_{post} + z^E_{pre}\)
        \item \(z^I_{post} + z^I_{pre}\)
        \item New synapses form when \alert{free} partner neurites are available.
        \item Synapses are deleted if neurites are \alert{retracted} by the neuron.
      \end{itemize}
    \end{column}
    \begin{column}{0.4\textwidth}
      \input{99_images/neuron-str-p.tex}
    \end{column}
  \end{columns}
\end{frame}
\begin{frame}[t]{Model core: activity dependent neurite growth (\(z\))}
  \vspace{0.4cm}
  \begin{columns}
    \begin{column}{0.5\textwidth}
      \centering
      \includegraphics[width=0.9\textwidth]{99_images/lipton1989.png}%chktex 8
    \end{column}
    \pause{}
    \begin{column}{0.5\textwidth}
      \centering
      \input{99_images/example-gaussian-1.tex}
    \end{column}
  \end{columns}
  \footnotetext[1]{\fullcite{Lipton1989}}
  \footnotetext[2]<2->{\fullcite{Butz2013}}
\end{frame}
\begin{frame}[t]{Growth curves: possibilities}
  \vspace{0.4cm}
  \begin{columns}
    \begin{column}{0.5\textwidth}
      \centering
      \input{99_images/example-gaussian-1.tex}
    \end{column}
    \begin{column}{0.5\textwidth}
      \centering
      \input{99_images/example-gaussian-2.tex}
    \end{column}
  \end{columns}
  \begin{center}
    We must determine 6 sets of growth curves: 3 for E, 3 for I neurons. Each growth curve has 4 free parameters.
  \end{center}
  \footnotetext[1]{\fullcite{Sinha2019a}}
\end{frame}
\begin{frame}[c]{Replicate peripheral lesion protocol}
  \begin{figure}[h]
    \centering
    \input{99_images/simulation-stages.tex}
  \end{figure}
\end{frame}
\section{Results and discussion}
\begin{frame}[c]{Deafferentation and successful repair}
  \begin{figure}
      \centering
      \resizebox{\textwidth}{!}{\input{99_images/201905131224-firing-rate-snapshots-E}}%
  \end{figure}
\end{frame}
\section{NeuroFedora: Free Software for Free Neuroscience}
